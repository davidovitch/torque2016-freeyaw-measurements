\documentclass[a4paper]{jpconf}
\usepackage[utf8]{inputenc}
\usepackage{graphicx}
\usepackage[hyphens]{url}

%\usepackage{citesort}
%\usepackage[square,sort&compress,numbers]{natbib}

%\bibliographystyle{plainnat}
%\bibliographystyle{vancouver}
%\bibliographystyle{natbib}
%\bibliographystyle{plainnat}
%\bibliographystyle{apasoft}

\bibliographystyle{iopart-num}
%\bibliographystyle{amsplain}
%\bibliographystyle{unsrt}

%\usepackage[backend=biber, style=numeric, citestyle=authortitle]{biblatex}
%\addbibresource{bibliography-iopart-num}

\begin{document}
\title{Open Access Wind Tunnel Measurements of a Downwind Free Yawing Wind Turbine}

\author{\textsuperscript{1} David Verelst, \textsuperscript{1} Torben Larsen and \textsuperscript{2} Jan-Willem van Wingerden}

\address{\textsuperscript{1} DTU Wind Energy - Loads and Control, \textsuperscript{2} TU Delft - Delft Center for Systems and Control}
	%\address{Loads And Control, DTU Wind Energy, Frederiksborgvej 399, 4000 Roskilde, Denmark}

\ead{dave@dtu.dk}

% max 200 words (approx)

%TODO either remove or add to proof the following:
% and that both blade sweep and rotor coning have a positive impact on free yaw stability

\begin{abstract}
A series of free yawing wind tunnel experiments was held in the Open Jet Facility (OJF) of the TU Delft. The $\approx 300$ W turbine has three blades in a downwind configuration and is optionally free to yaw. Different 1.6m diameter rotor configurations are tested such as blade flexibility and sweep. This paper gives a brief overview of the measurement setup and challenges, and continues with presenting some key results. This wind tunnel campaign has shown that a three bladed downwind wind turbine can operate in a stable fashion under a minimal yaw error. Finally, a description of how to obtain this open access dataset, including the post-processing scripts and procedures, is made available via a publicly accessible website.
\end{abstract}


% =================================================================================
\section{Introduction}

The main focus of this publication is to present global performance parameters (thrust, free yaw response and indicative mechanical power/torque estimates) of results from a wind turbine measurement campaign of a free yawing downwind 3 bladed wind turbine ($\approx 300$ W, 1.6m rotor diameter). These results are published as an open access data set, and this paper intents to serve as its formally peer reviewed reference. The wind tunnel measurements were completed within the context of a PhD research project
\cite{verelst_numerical_2013:diss}, and some of the preliminary results have been discussed earlier in comparison with aeroelastic simulations
\cite{verelst_wind_2014}. This paper further complements the earlier publications with additional free yawing results and indicative power/torque estimates. However, the presented work is not complete and not all issues concerning the presented data set are resolved. %Several compromises had to be made due to numerous challenges when preparing for these experiments, and this resulted in shortcoming and limitations which are discussed in the following section.

%Additionally, all the measurement data is now also made available as an open access dataset and this paper should serve as a formal, peer-reviewed reference for these measurements. The open access dataset includes both raw and corrected/calibrated data. Finally, the scripts that have been used for the correction and calibration procedures are provided. In doing so, other researchers can review the entire cycle from raw measurement to coherent and calibrated data set.

%TODO literature review
% ---------------------FROM ABSTRACT
%A short and incomplete list of other yawed inflow related experiments and measurements will be briefly discussed in the full paper. Although extensive measurement campaigns have been undertaken previously regarding yawed inflow conditions, to the author knowledges there are no other existing experiments that have studied the yaw stability of three bladed downwind free yawing wind turbines.
% ----------------------------------

Previous rotating wind tunnel experiments with at least a partial focus on yawed inflow conditions have been discussed in for example:
\cite{haans_measurement_2005}, \cite{haans_measurement_2005-1}, \cite{bracchi_downwind_2014}, \cite{schepers_engineering_2012}, \cite{schepers_final_2012}, \cite{mexnext_iea_web}, \cite{schepers_model_2007}, \cite{hand_unsteady_2001}, \cite{loland_wind_2011}, \cite{haans_wind_2011}.
Although \cite{bracchi_downwind_2014} considers a downwind rotor design, it does not explore the free yawing concept. These are an extensive set of measurement campaigns regarding yawed inflow conditions. However, to the author knowledges there are no other existing experiments that have studied the yaw stability of three bladed downwind free yawing wind turbine.

A 3 bladed free yawing downwind wind turbine could potentially reduce the yaw drive torque requirements, minimize its wear, maintenance, and hence ultimately reduce its costs. Assuming such a free yawing system can operate in a stable fashion and under minimal yaw error, it might also reduce average operating yaw errors compared to actively controlled yaw systems. In doing so the turbine loading could decrease while power output can be increased. Active yaw controllers have slow response times (in the order of minutes) and are based on low accuracy wind direction measurements (often installed on the nacelle directly behind the rotor). However, the blade tower shadow passage of a downwind wind turbine introduces a challenge due to increased blade fatigue loading and reduced power output.
%Free yawing downwind wind turbines are not a new concept. Currently they are used only on either very small wind turbines (W-kW range) or small to medium sized machines (50-100 kW range). To the authors knowledge, there are no MW machines in operation today utilizing the downwind free yawing concept. Large wind turbines will likely to always require an active yawing mechanism (required for periodic cable unwinding), however, a free yawing turbine could potentially reduce the yaw drive torque requirements and minimize its wear and maintenance.

An alternative application of a good understanding of wind turbine performance and loading under yawed inflow could be to help predict more precisely the wind direction \cite{bottasso_validation_2015}. Considering this is often problematic when using only aerodynamic devices placed on the nacelle, a reliable method to determine the yaw error based on turbine loading and/or performance parameters can be valuable.


% =================================================================================
\section{Experimental Setup and Limitations}

% ------------ FROM THE ABSTRACT
%A short description of the experimental setup will be given in the full paper, with a focus on the available sensors and the limitations on accuracy of the collected data. Further, this section will list some valuable lessons learned from running a wind tunnel campaign on a small budget and team.
% ------------------------------

The experiment consisted out of two campaigns: the first took place during February of 2012, and the second two months later in April 2012. A comprehensive discussion of the design of the experimental setup and measurement sensors are given in sections 4.4, 4.5 and 5.2 of \cite{verelst_numerical_2013:diss}. What follows is a brief overview:

\begin{figure}[h]
\begin{minipage}{17pc}
\includegraphics[width=17pc]{figures/model/aero-blade-layout-hawtopt_blade.dat.eps}
\caption{\label{rotor-aero-design} Rotor aerodynamic design.}
\end{minipage}\hspace{3pc}%
\begin{minipage}{17pc}
\vspace{10px}
\includegraphics[width=17pc]{figures/model/h2-blade-stiffness-flex-vs-stiff.eps}
\caption{\label{h2-blade-stiffness} Estimated blade flapwise stiffness distribution.}
\end{minipage} 
\end{figure}

\begin{table}
\centering
\begin{tabular}{c c c c c c}
\hline
          & region & $t/c$ & $Re_{design}$ & $Re_{data}$ & $C_{L_{max}}$ \\
NREL S823 & inboard  & $21\%$ & $4e5$      & $1e5$      & $1.184$ \\
NREL S822 & outboard & $16\%$ & $6e5$      & $2e5$      & $1.100$ \\
\hline
\end{tabular}
\caption{Blade airfoils and corresponding key parameters.}
\label{table:S822_S823_keyproperties}
\end{table}

\begin{itemize}
	\item Rotor aerodynamic performance (ignoring flexibility): 280 W at 450 RPM and 11.4 m/s, $C_{P_{max}}=0.36$ at TSR=6 (tip speed ratio).
	\item Rotor diameter: 1.60m, blade root radius : 0.245m, blade length=0.555m.
	\item The tower base is suspended on two bearings, allowing the complete tower to yaw. The nacelle is fixed to the tower top.
	\item Blades are made from PVC foam, optionally reinforced with glass fibre sandwich beam. Various rotor configurations have been considered (but all share the same aerofoil, and twist, chord length, and aerofoil thickness distributions, see figure \ref{rotor-aero-design}):
	\begin{itemize}
		\item Blade aerofoils: NREL S823 and S822 \cite{somers_s822_2005} (see table \ref{table:S822_S823_keyproperties}). Aerodynamic characteristics are taken from the UIUC LSAT database \cite{selig_summary_1996}.
		\item All blades are made from liquid PVC foam that was poured into a negative glass fibre mould. Some blades had an additional glass fibre sandwich core. Surface roughness was very smooth, expect for a higher roughness area along the leading edge ($\approx 1$ mm wide).
		\item Blade set A "stiff": 8-layered glass fibre sandwich core beam, strain gauges in flapwise direction (glued on core beam), flapwise stiffness derived from measured blade deflection curves with static tip load (see figure \ref{h2-blade-stiffness}).
		\item Blade set B "flexible": 2-layered glass fibre sandwich core beam, strain gauges in flapwise direction (glued on core beam), flapwise stiffness derived from measured blade deflection curves with static tip load (see figure \ref{h2-blade-stiffness}).
		\item Blade set C "flexible-no-beam": only PVC foam and no strain gauges, stiffness not quantified.
		\item Blade set D "flexible-no-beam-sweep": swept planform, only PVC foam and no strain gauges, stiffness not quantified.
	\end{itemize}
	\item In free yawing mode the yaw angle range is approximately -35 to 35 degrees. The yaw angle can be locked or manually controlled in free yaw mode from the control room using a system of simple ropes and pulleys.
	\item Positive values for the yaw angle $\psi$ mean that blade moving upwards is closer to the wind. Alternatively, the yaw axis rotation vector is pointing down (vertically), and the rotor axis rotation vector is pointing into the wind (meaning the rotor is rotating "the wrong way").
	\item The generator is connected to a high and low electrical resistance dump load for limited torque variability (there is no active rotor speed control). The effective dump load magnitude is controlled by encoding a given set point with pulse width modulation (very fast switching between either dump loads). The low resistance is fully active for a duty cycle (dc) of 1, and the high resistance for a duty cycle value of 0. Any value in between can be considered as an interpolation between the high and low resistance dump load.
	\item Main measurement channels:
	\begin{itemize}
		\item Electrical power dissipated in the dump load.	
		\item Rotor speed and azimuth angle measurement device.
		\item Strain gauges on the tower base in for-aft and side-side directions.
		\item Strain gauges on two blades of set A "stiff" and set B "flexible" in flapwise direction.
		\item Laser distance meter to measure the yaw angle.
	\end{itemize}
%	\item 3D-accelerometer at the tower top.
%	\item 2 data acquisition systems (DAQ) for data recording: blade strain gauges are connected to a wireless DAQ, all other channels connected to a wired DAQ. A shared reference signal on both the wired and wireless DAQ system is used to synchronise both
%	\item Wind speed, temperature and static pressure measurements were taken from the  OJF measurement system.
\end{itemize}

The limitations of the experiment are summarised as follows:

\begin{itemize}
	\item Reliable measurement of the mechanical shaft power is missing.
	\item No yaw moment measurement device.
	\item Blade flap-wise strain gauge measurements are not reliable and significantly affected by centrifugal forces.
	\item Smaller blade deflections than aimed for due to higher than expected blade mass and PVC foam stiffness. This means
	\item Mass imbalances, especially for blade sets A "stiff" and B "flexible".
	\item Not so accurate pitch settings ($\pm 1$ deg), small pitch and cone angle imbalance.
	\item The results from the February campaign are largely unusable due to corrupted tower strain calibration measurements, low quality rotor speed measurements, low generator torque range and lacking generator data sheet (generator was upgraded in April).
	\item No accurate aerodynamic performance characteristics of the rotor (3D-correct lift, drag and moment coefficients).
\end{itemize}

However, the missing shaft torque/power measurement can be compensated to a certain extent by considering the measured torque-rpm-power characteristics provided by the generator manufacturer. This is discussed in more detail in the following sections.


% =================================================================================
\section{Measurement Results}

\subsection{Overview}

An overall overview of the test cases can be obtained by considering figures \ref{rpm-vs-wind} and \ref{ct-vs-lambda}. From figure \ref{rpm-vs-wind} the two operating modes of the turbine are clearly illustrated: low tip speed ratios when the rotor is operating in deep stall, and higher tip speed ratios (around the design point) in attached flow regimes. 

\begin{figure}[h]
\begin{minipage}{17pc}
\vspace{10px}
\includegraphics[width=17pc]{figures/overview/symlinks_all-rpm-vs-wind-dc-all}
\caption{\label{rpm-vs-wind} Rotor speed as function of wind speed in aligned flow for various generator load settings (dc). Dotted lines indicate tip speed ratios.}
\end{minipage}\hspace{3pc}%
\begin{minipage}{17pc}
\includegraphics[width=17pc]{figures/overview/symlinks_all-ct-vs-lambda-april-blades-straight}
\caption{\label{ct-vs-lambda} Thrust coefficients as function of tip speed ratio in aligned flow for various wind speeds.}
\end{minipage} 
\end{figure}


The influence of yaw inflow angle on the rotor thrust is given for different tip speed ratios in figures \ref{ct-vs-yawerror-straight} and \ref{ct-vs-yawerror-swept} for the straight and swept blades respectively. The dash line indicated in both figures show that there is an indicative $cos^2 \psi$ (with $\psi$ being the yaw error) relationship between thrust and yaw angle. Further, a small skew in figures \ref{ct-vs-yawerror-straight} and \ref{ct-vs-yawerror-swept} is noted: the thrust coefficient has a slight asymmetry. For negative yaw errors the $C_T$ is slightly higher compared to positive ones. Although there are less data points available for the swept blade, these results indicate they perform similarly. A definitive explanation for this asymmetry remains to be formulated. Possible causes could be due to tower shadow effects, misalignments in thrust measurements (which are based on for-aft and side-side strain gauges), or hypothetical non uniform artefacts in the wind tunnel velocity distribution. Investigation this further is referred to future work.

%TODO make sure the cos2 lines are the same for both straight and swept blade plots

\begin{figure}[h]
\begin{minipage}{17pc}
%\vspace{10px}
\includegraphics[width=17pc]{figures/overview/symlinks_all-yawerror-vs-ct-april-straight-blades}
\caption{\label{ct-vs-yawerror-straight} Thrust coefficients as function of yaw angle for various tip speed ratios and straight blades (sets A and B). Dashed lines are proportional to $cos^2 \psi$ and are the same as in figure \ref{ct-vs-yawerror-swept} to facilitate visual inspection.}
\end{minipage}\hspace{3pc}%
\begin{minipage}{17pc}
\includegraphics[width=17pc]{figures/overview/symlinks_all-yawerror-vs-ct-april-swept-blades}
\caption{\label{ct-vs-yawerror-swept} Thrust coefficients as function of yaw angle for various tip speed ratios and swept blades (set D). Dashed lines are proportional to $cos^2 \psi$ and are the same as in figure \ref{ct-vs-yawerror-straight} to facilitate visual inspection.}
\end{minipage} 
\end{figure}


% -------------------------------------------------------------------------------
\subsection{Estimating Rotor Mechanical Power}

The mechanical and electrical performance of the generator have been documented by the manufacturer in the form of a rotor speed vs input torque, electrical current and voltage, and are given for a range of generator load resistance values in figure \ref{rpm2torque-windbluepower-resistance}. The contour lines in figure \ref{rpm2torque-windbluepower-resistance} indicate regions for which the resistance is constant, and the labels indicate the magnitude of this resistance ranging from $28 \Omega$ (units Ohm, in dark red) to $4 \Omega$ (units Ohm, in dark blue). The corresponding generator efficiency is given in figure \ref{rpm2torque-windbluepower-eff}. Note that the efficiency only exceeds 56\% for rotor speeds above 400 RPM. 

\begin{figure}[h]
\centering
\begin{minipage}{18pc}
\centering
\includegraphics[width=\textwidth]{figures/generator-st-540-contour}
\caption{\label{rpm2torque-windbluepower-resistance} Measured applied torque and rotor speed for various electrical load settings (contour labels units are in Ohm).}
\end{minipage}\hspace{1pc}%
\begin{minipage}{18pc}
\centering
\includegraphics[width=\textwidth]{figures/generator-st-540-contour-eff}
\caption{\label{rpm2torque-windbluepower-eff} Measured generator electrical efficiency for a range of applied torque and rotor speeds at various electrical load settings (contour labels in \%).}
\end{minipage}
\end{figure}

The measurement data given in figures \ref{rpm2torque-windbluepower-resistance} and \ref{rpm2torque-windbluepower-eff} is provided by generator manufacturer windbluepower.com and is re-used with their permission.

Although the effective generator load resistance was not determined accurately for the experimental setup, an estimate is available: $11 \Omega$ ([Ohm]) for duty cycle dc=1, and $28 \Omega$ for dc=0. Based on this data set, the generator electrical efficiency, mechanical torque and electrical power can be estimated by simply using the data of figure \ref{rpm2torque-windbluepower-resistance} as a lookup table: at a given generator speed and dump load magnitude (given by the duty cycle value) the mechanical torque can be found. The rotor speed measurement can now be used to estimate the rotor mechanical power and the generator electrical power. Note that due to a limited range of rotor speed data points in the generator measurement sheet provided by the manufacturer, not all wind tunnel measurement points have a corresponding estimated mechanical/electrical power value.

A comparison between the measured electrical power dissipated in the dump load and the estimated mechanical power (based on manufacturer data sheet) is given in figures \ref{pe-meas-vs-lambda} and \ref{pe-rpm2torque-vs-lambda} respectively. Both the measured and the estimated values correspond in order of magnitude. However, significant differences are noted:
\begin{itemize}
	\item Measured electrical power (figure \ref{pe-meas-vs-lambda}) seems to indicate a parabolic dependency on rotor speed, where the peak power nearly approaches the estimated mechanical power values as given in figure \ref{pe-rpm2torque-vs-lambda}. This indicates that significant electrical losses occur elsewhere in the system, which can be confirmed by considering figure \ref{rpm2torque-windbluepower-eff} that shows rapidly decreasing efficiency with decreasing rotor speed (or decreasing TSR at a given wind speed).
	\item Figure \ref{pe-meas-vs-dc} shows that the measured electrical power rapidly drops for duty cycles approaching 1 (low resistance dump load setting). The corresponding mechanical power estimates in figure \ref{pe-rpm2torque-vs-dc} show only a modest decrease in power when approaching duty cycle settings of 1.
	\item An more accurate and in detailed account of these electrical power measurements and corresponding estimated mechanical values is referred to future work.
\end{itemize}


\begin{figure}[h]
\begin{minipage}{17pc}
\includegraphics[width=17pc]{figures/overview/symlinks_all-pe-meas-vs-lambda-april-blades-straight.eps}
\caption{\label{pe-meas-vs-lambda} Measured electrical power as function of tip speed ratio in aligned flow for various wind speeds. Blade sets A and B.}
\vspace{10px}
\end{minipage}\hspace{3pc}%
\begin{minipage}{17pc}
\includegraphics[width=17pc]{figures/overview/symlinks_all-pe-rpm2torque-vs-lambda-april-blades-straight.eps}
\caption{\label{pe-rpm2torque-vs-lambda} Generator electrical power based on RPM-torque data provided by manufacturer, using estimate for resistance value of dump load. Blade sets A and B.}
\end{minipage} 
\end{figure}


\begin{figure}[h]
\begin{minipage}{17pc}
\includegraphics[width=17pc]{figures/overview/symlinks_all-pe-measured-vs-dc-all.eps}
\caption{\label{pe-meas-vs-dc} Measured electrical power as function of generator dump load resistance magnitude. Blade sets A and B.}
\vspace{10px}
\end{minipage}\hspace{3pc}%
\begin{minipage}{17pc}
\includegraphics[width=17pc]{figures/overview/symlinks_all-pe-rpm2torque-vs-dc-all.eps}
\caption{\label{pe-rpm2torque-vs-dc} Generator electrical power (based on RPM-torque estimates) as function of generator dump load magnitude.}
\end{minipage} 
\end{figure}


Assuming that these estimate generator characteristics are somewhat sensible, the rotor power coefficient $C_P$ (based on the estimated mechanical power) can be considered as function of TSR and yaw error in figures \ref{cp-rpm2torque-vs-lambda-straight} and \ref{cp-rpm2torque-vs-yawerror-straight}. However, figure \ref{cp-rpm2torque-vs-lambda-straight} seems to indicate that the power coefficient has a Reynolds number dependency, while previously the thrust coefficients in figure \ref{ct-vs-lambda} did not. Due to several possible sources of errors in the power estimation scheme considered here, it is likely that the dump load resistance estimates are inaccurate. It is suggested that a more detailed consideration and modelling of the electrical system is required (including pulse width modulation dump load switch) to increase the reliability of the mechanical power estimate. For example, the measured generator efficiency (figure \ref{rpm2torque-windbluepower-eff}) could be used to improve the effective generator dump load magnitude. The latter is also affected by power dissipated in the pulse width modulation system (which required active cooling using a fan and aluminium cooling block when operating at duty cycles other than 0 and 1) and the wiring between the generator and the dump load. However, these losses were not quantified during the experiment.

Due to the inaccuracies related with the estimated $C_P$, figure \ref{cp-rpm2torque-vs-yawerror-straight} is only included here as an indicative measure of yaw error on power coefficient. However, it is interesting to note that a similar (unexplained) asymmetry with respect to yaw error is observed compared to the thrust coefficients (figure \ref{ct-vs-yawerror-straight}): negative yaw errors result in slightly higher power coefficients.

%\clearpage

\begin{figure}[h]
\begin{minipage}{17pc}
\includegraphics[width=17pc]{figures/overview/symlinks_all-cp-rpm2torque-vs-lambda-april-blades-straight}
\caption{\label{cp-rpm2torque-vs-lambda-straight} Estimated power coefficients as function of tip speed ratio in aligned flow for various wind speeds. Blade sets A and B.}
\vspace{10px}
\end{minipage}\hspace{3pc}%
\begin{minipage}{17pc}
\includegraphics[width=17pc]{figures/overview/symlinks_all-yawerror-vs-cp-rpm2torque-april-straight-blades}
\caption{\label{cp-rpm2torque-vs-yawerror-straight} Estimated power coefficients as function of yaw angle for various tip speed ratios and straight blades. Dashed lines are proportional to $cos^3 \psi$. Blade sets A and B.}
\end{minipage} 
\end{figure}


% -------------------------------------------------------------------------------
\subsection{Free yaw response}

For the free yawing tests, the turbine was forced into a yaw error and released again after reaching steady rotor speed conditions. An example of such a result is given in figure \ref{freeyaw-flex-vs-samo}: the rotor speed and yaw inflow angle are given for two different blade planform layouts: straight (blade sets A and B) and swept (blade set D). It is interesting to note that the yaw response is either under- or critically-damped. The rotor speed response is always critically- or over-damped. The largest contributing factor for the type of yaw response (critically- or under-damped) depends on the operating tip speed ratio, initial yaw error (both magnitude and sign). It is assumed this is due to the yaw moment dependency of rotor speed, wind speed and yaw inflow angle. A quantitative comparison between the performance of the free yaw response of the swept (set D) and straight blades (set A or B) would require the same initial conditions for both rotor speed and yaw angle. Unfortunately, such a combination is not available in the current data set.

\begin{figure}[h]
\centering
\begin{minipage}{\textwidth}
\centering
\includegraphics[width=25pc]{figures/freeyaw/277-vs-330-9ms-rpm.eps}
%\caption{\label{freeyaw-flex}Comparison .}
\end{minipage}
\begin{minipage}{\textwidth}
\centering
\includegraphics[width=25pc]{figures/freeyaw/277-vs-330-9ms-yaw.eps}
\caption{\label{freeyaw-flex-vs-samo} Comparison of a typical free yaw response of a straight and flexible blade at 9 m/s. Note the overshoot of the swept blade (set D) is assumed to be due to the higher initial yaw error and consequently higher initial yawing moment.}
\end{minipage} 
\end{figure}

A series of these free yaw response experiments were recorded at various wind and rotor speeds, but unfortunately not at exactly the same initial yaw error and rotor speed. This makes a direct comparison difficult, as can be seen from figures \ref{allfreeyaw-respons-rpm} and \ref{allfreeyaw-respons-yaw} showing the respective rotor speed and yaw angle time traces. Figures \ref{allfreeyaw-respons-rpm}-\ref{allfreeyaw-respons-yaw-norm} indicate:
\begin{itemize}
	\item red for positive initial yaw errors
	\item blue for negative initial yaw errors
	\item triangles refer to the swept blades
	\item lines refer to straight blades
	\item figure \ref{allfreeyaw-respons-yaw} has a reversed left y-axis (yaw)
\end{itemize}

Note that for positive yaw errors the blade is moving upwards when rotated into the wind.

In order to perform a qualitative comparison between the different free yaw responses, and in an attempt to establish if there is a dependency on blade sweep (blade set D), the free yaw response rotor speed and yaw angles are normalized in figures \ref{allfreeyaw-respons-rpm-norm} and \ref{allfreeyaw-respons-yaw-norm}.

The normalized rotor speed response (figure \ref{allfreeyaw-respons-rpm-norm}) seems roughly independent of blade type and initial yaw error. This indicates that the rotor torque changes due yaw misalignment are qualitatively speaking similar cases presented here.

\begin{figure}[h]
\begin{minipage}{\textwidth}
\centering
\includegraphics[width=33pc]{figures/freeyaw/allfreeyaw_rpm.eps}
\caption{\label{allfreeyaw-respons-rpm} Free yaw response: rotor speed [rpm]. Red marks positive initial yaw errors, blue negative. Series indicated with a triangle refer to the swept blades, lines to straight blades.}
\end{minipage}
\begin{minipage}{\textwidth}
\centering
\includegraphics[width=33pc]{figures/freeyaw/allfreeyaw_yaw_angle.eps}
\caption{\label{allfreeyaw-respons-yaw} Free yaw response: yaw angle [deg]. Red marks positive initial yaw errors, blue negative. Series indicated with a triangle refer to the swept blades, lines to straight blades. Left y-axis is reversed compared to right y-axis.}
\end{minipage} 
\end{figure}

When considering the yaw angle response (figures \ref{allfreeyaw-respons-yaw} and \ref{allfreeyaw-respons-yaw-norm}) the following observations are made:

\begin{itemize}
	\item The steady state yaw angle is not exactly zero, and depends on the initial yaw error (figure \ref{allfreeyaw-respons-yaw}). It is possible that the yaw inertia and bearing friction is of similar magnitude and therefore the steady yaw angle gets "stuck" in a region of low yawing moments. Detailed yaw moment measurements and/or simulations are required to further investigate this issue.
	\item Negative initial yaw errors result in a marginally higher steady state yaw angles (figure \ref{allfreeyaw-respons-yaw}). A simple explanation could be that there is very small error on the yaw alignment. Alternatively, the tower shadow could have a contribution towards an asymmetric yaw inflow vs yaw moment distribution, and possibly a non zero yaw angle for zero yawing moment.
	\item A qualitative indication of the yaw stiffness is derived based on the slope of the free yaw response. Yaw stiffness for negative yaw errors seems to be lower compared to positive ones. Corresponding to the higher yaw stiffness, only positive yaw errors demonstrate an under-damped response for some cases.
%	\item The swept blade performs similarly compared to the straight one, although it seems to indicates a modest increased yaw stiffness. However, this should be investigated further in more detail before forming a more conclusive opinion.
\end{itemize}


\begin{figure}[h]
\begin{minipage}{\textwidth}
\centering
\includegraphics[width=33pc]{figures/freeyaw/allfreeyaw_rpm_norm.eps}
\caption{\label{allfreeyaw-respons-rpm-norm} Normalized free yaw response: rotor speed.}
\end{minipage}
\begin{minipage}{\textwidth}
\centering
\includegraphics[width=33pc]{figures/freeyaw/allfreeyaw_yaw_angle_norm.eps}
\caption{\label{allfreeyaw-respons-yaw-norm} Normalized free yaw response: yaw angle.}
\end{minipage} 
\end{figure}

%\clearpage

% =================================================================================
\section{Open Access Measurement Data}

The underlying data set of this paper is open access and can be freely downloaded:
\begin{itemize}
	\item Python post-processing callibration methods, plotting scripts, and representative aeroelastic beam model in HAWC2 format \cite{github:freeyaw-ojf-wt-tests}. This serves at the main access points and includes further instructions on how to access the measurement data and contains descriptions on the data formats used.
	\item Measurement results (raw and calibrated), figures and plots of the results, pictures and video's of the experiment \cite{deic:data-sources}
	\item \LaTeX sources and figures for this paper \cite{github:torque2016-freeyaw-measurements}
\end{itemize}

This paper only presents a part of the available measurements, and the interested reader is referred to the given websites for more results.

%Making datasets like this publicly available including documentation and instructions to reproduce (parts) of the analysis is not often considered due to various reasons. It is a significant additional effort to sanitize the data and post-processing scripts in order to make it publicly available. This is necessary though if the data is mend to be useful for other users or researchers. %Considering that creating and post-processing experimental datasets like these represent a considerable effort, it is worthwhile to also share the data with as many other researchers as possible.

%Since creating and post-processing experimental data is 

% =================================================================================
\section{Future Work, Recommendations Future Experiments}

The estimated mechanical power results are not sufficiently accurate, and some suggestions were made that could help to improve it. Further, the slightly higher thrust coefficients at negative yaw errors need to be considered in more detail.

The current results have not been compared in detail with other yawed inflow experiments reported in literature, and this should be considered in future discussions in order to verify or refute conclusions drawn based on the this data set.

Some of the drawbacks of this experiment, such as missing accurate aerodynamic performance characteristics of the rotor (lift, drag and moment coefficients), can be mitigated by considering a large number of operating conditions. For example, in \cite{bottasso_calibration_2014} the aerodynamic rotor characteristics are determined based on a large number of measurements with varying tip speed ratios and blade pitch angles. This requires careful planning of the experiment upfront the measurement campaign. 

Fixed rotor speed control is essential when trying to cover a broad range of operating conditions. In doing so, it would be possible to test a larger range of tip speed ratios at various yawed inflow angles, and ideally also at various blade pitch angle settings.

When using a simple direct drive generator without active control, the outcome of the experiment would benefit from its accurate characterisation. Not only the torque-rpm-electrical power characteristics, but also the performance under various electrical loads, and the dynamic response.

So far the presence of a yawing wind turbine in the wind tunnel is assumed to not have any effect on the inflow conditions. Since the considered wind tunnel has an open measurement section, wall and blocking effects have been assumed to be insignificant. A more detailed aerodynamic investigation is required in order to verify that claim, but this is considered out of the scope here. %At the time of the experiments, the authors did not have access to detailed aerodynamic studies regarding the performance of the TU Delft Open Jet Facility.

Finally, the yaw stability could be studied in more detail if a combination of various free yaw responses is combined with fixed yaw moment measurements, and a quantification of the yaw bearing friction moment.

% =================================================================================
\section{Conclusions}

This paper presented a short overview of some of the key results of a wind tunnel campaign of a free yawing down wind turbine, and illustrated that it can operate in a steady fashion under minimal yaw error in a free yawing configuration. Thrust and estimated power coefficients at various tip speed ratios have been presented for various yaw angles. However, the estimated mechanical power is not sufficiently accurate and a more detailed model of the electrical system has to be considered in order to improve it. In yawed inflow conditions the thrust is proportional to $cos^2 \psi$, and a small asymmetry is observed resulting in slightly higher thrust values for negative yaw errors compared to positive ones is noted, but a definitive explanation is not given. In free yaw the turbine operates in a stable fashion, and the system has typically a critically-damped response when released from a larger (30-25 degrees) yaw error. Instructions are provided in order to download this open access data set.


% =================================================================================
\section*{References}

\begingroup
\raggedright
%\bibliographystyle{iopart-num}
\bibliography{bibliography-iopart-num}

%\printbibliography

%\begin{thebibliography}{9}
%
%\bibitem{Erdos01} P. Erd\H os, \emph{A selection of problems and
%results in combinatorics}, Recent trends in combinatorics (Matrahaza,
%1995), Cambridge Univ. Press, Cambridge, 2001, pp. 1--6.
%
%\bibitem{ConcreteMath}
%R.L. Graham, D.E. Knuth, and O. Patashnik, \emph{Concrete
%mathematics}, Addison-Wesley, Reading, MA, 1989.
%
%\bibitem{Knuth92} D.E. Knuth, \emph{Two notes on notation}, Amer.
%Math. Monthly \textbf{99} (1992), 403--422.
%
%\bibitem{Simpson} H. Simpson, \emph{Proof of the Riemann
%Hypothesis},  preprint (2003), available at 
%\url{http://www.math.drofnats.edu/riemann.ps}.
%
%\end{thebibliography}

\endgroup

\end{document}
